\section{Kody i kaflowania - Kody Hamminga: konstrukcja; własności korygujące.}

\textbf{Odległość Hamminga} - wprowadzona przez Richarda Hamminga miara odmienności dwóch ciągów o takiej samej długości,
wyrażająca liczbę miejsc (pozycji), na których te dwa ciągi się różnią.
Innymi słowy jest to najmniejsza liczba zmian (operacji zastępowania elementu innym),
jakie pozwalają przeprowadzić jeden ciąg na drugi.

Przykłady:
\begin{itemize}
    \item odległość pomiędzy ciągami 10011101 i 10\textbf{1}11\textbf{0}01 wynosi 2,
    \item odległość pomiędzy ciągami zagrabić i za\textbf{t}r\textbf{ą}bi\textbf{ł} wynosi 3.
\end{itemize}

\textbf{Kod Hamminga} - liniowy kod korekcyjny.
Kod Hamminga wykrywa i koryguje błędy polegające na przekłamaniu jednego bitu (single-error correction).
Kod ten może wykrywać (ale już nie korygować) błędy podwójne (dwa jednocześnie przekłamane bity - double-error detection).
Jest to możliwe, gdy wykorzystany zostanie dodatkowy bit parzystości.

Dla porównania: prosty kod z kontrolą parzystości nie może korygować żadnych błędów
ani też nie może być używany do detekcji błędu na więcej niż jednym bicie.

\begin{tabular}{ |c|c|c|c|c|c|c|c|c|c|c| }
    \hline
    Pozycja bitu                    & 1     & 2     & 3     & 4     & 5     & 6     & 7     \\
    \hline

    Bit parzystości (p), danych (d) & $p_1$ & $p_2$ & $d_1$ & $p_4$ & $d_2$ & $d_3$ & $d_4$ \\
    \hline

    p1                              & ×     &       & ×     &       & ×     &       & ×     \\
    p2                              &       & ×     & ×     &       &       & ×     & ×     \\
    p4                              &       &       &       & ×     & ×     & ×     & ×     \\
    \hline
\end{tabular}\\

W roku 1950 Hamming przedstawił kod $(7,4)$, który kodował 4 pozycje informacyjne jako słowo 7-bitowe,
dodając 3 bity parzystości.
Macierz generująca $G$ kodu $(7,4)$ i jego macierz kontroli parzystości $H$ są przedstawione poniżej:


\begin{align*}
    G^{T} \coloneqq \begin{pmatrix}
                        1 & 1 & 0 & 1 \\
                        1 & 0 & 1 & 1 \\
                        1 & 0 & 0 & 0 \\
                        0 & 1 & 1 & 1 \\
                        0 & 1 & 0 & 0 \\
                        0 & 0 & 1 & 0 \\
                        0 & 0 & 0 & 1
    \end{pmatrix} & &
    H \coloneqq \begin{pmatrix}
                    1 & 0 & 1 & 0 & 1 & 0 & 1 \\
                    0 & 1 & 1 & 0 & 0 & 1 & 1 \\
                    0 & 0 & 0 & 1 & 1 & 1 & 1
    \end{pmatrix}
\end{align*}

\subsection{Przykład}
Chcemy przetransmitować $1011$.
\begin{align*}
    p=\begin{pmatrix}
          d_1 \\
          d_2 \\
          d_3 \\
          d_4
    \end{pmatrix} = \begin{pmatrix}
                        1 \\
                        0 \\
                        1 \\
                        1
    \end{pmatrix}
\end{align*}
\begin{align*}
    x=G^{T}p=\begin{pmatrix}
                 1 & 1 & 0 & 1 \\
                 1 & 0 & 1 & 1 \\
                 1 & 0 & 0 & 0 \\
                 0 & 1 & 1 & 1 \\
                 0 & 1 & 0 & 0 \\
                 0 & 0 & 1 & 0 \\
                 0 & 0 & 0 & 1
    \end{pmatrix}
    \begin{pmatrix}
        1 \\
        0 \\
        1 \\
        1
    \end{pmatrix}=
    \begin{pmatrix}
        2 \\
        3 \\
        1 \\
        2 \\
        0 \\
        1 \\
        1
    \end{pmatrix}=
    \begin{pmatrix}
        0 \\
        1 \\
        1 \\
        0 \\
        0 \\
        1 \\
        1
    \end{pmatrix}
\end{align*}

Zatem $0110011$ jest kodem Hamminga odpowiadającym wiadomości $1011$.

Sprawdzamy parzystość.
\begin{align*}
    z=Hr=\begin{pmatrix}
             1 & 0 & 1 & 0 & 1 & 0 & 1 \\
             0 & 1 & 1 & 0 & 0 & 1 & 1 \\
             0 & 0 & 0 & 1 & 1 & 1 & 1
    \end{pmatrix}
    \begin{pmatrix}
        0 \\
        1 \\
        1 \\
        0 \\
        0 \\
        1 \\
        1
    \end{pmatrix}=
    \begin{pmatrix}
        2 \\
        4 \\
        2
    \end{pmatrix}=
    \begin{pmatrix}
        0 \\
        0 \\
        0
    \end{pmatrix}
\end{align*}
Wynikiem jest wektor zerowy - odbiorca może stwierdzić, że nie doszło do przekłamania.

W przeciwnym wypadku występuje korekcja błędu.
\[
    r = x + e_i
\]
gdzie $e_i$ to wektor zerowy z jedynką na pozycji $i$.
\[
    Hr=H(x+e_i)=Hx+He_i
\]
Zauważmy, że x to oryginalna wiadomość, stąd $Hx=\mathbf{0}$.
\[
    r=x+e_5=
    \begin{pmatrix}
        0 \\
        1 \\
        1 \\
        0 \\
        0 \\
        1 \\
        1
    \end{pmatrix}+
    \begin{pmatrix}
        0 \\
        0 \\
        0 \\
        0 \\
        1 \\
        0 \\
        0
    \end{pmatrix}=
    \begin{pmatrix}
        0 \\
        1 \\
        1 \\
        0 \\
        1 \\
        1 \\
        1
    \end{pmatrix}
\]
\begin{align*}
    z=Hr=\begin{pmatrix}
             1 & 0 & 1 & 0 & 1 & 0 & 1 \\
             0 & 1 & 1 & 0 & 0 & 1 & 1 \\
             0 & 0 & 0 & 1 & 1 & 1 & 1
    \end{pmatrix}
    \begin{pmatrix}
        0 \\
        1 \\
        1 \\
        0 \\
        1 \\
        1 \\
        1
    \end{pmatrix}=
    \begin{pmatrix}
        3 \\
        4 \\
        3
    \end{pmatrix}=
    \begin{pmatrix}
        1 \\
        0 \\
        1
    \end{pmatrix}
\end{align*}
\textbf{Syndrom} $101$ odpowiada wartości 5 - stąd wnioskujemy, że piąty bit jest przekłamany.
\[
    r_{corrected}=\begin{pmatrix}
                      0       \\
                      1       \\
                      1       \\
                      0       \\
                      \bar{1} \\
                      1       \\
                      1
    \end{pmatrix}=\begin{pmatrix}
                      0 \\
                      1 \\
                      1 \\
                      0 \\
                      0 \\
                      1 \\
                      1
    \end{pmatrix}
\]
Dekodowanie. Najpierw, definiujemy macierz $R$.
\begin{align*}
    R=&\begin{pmatrix}
           0 & 0 & 1 & 0 & 0 & 0 & 0 \\
           1 & 0 & 0 & 0 & 1 & 0 & 0 \\
           1 & 0 & 0 & 0 & 0 & 1 & 0 \\
           0 & 0 & 0 & 0 & 0 & 0 & 1
    \end{pmatrix}
\end{align*}
Dalej obliczamy otrzymaną wartość.
\begin{align*}
    p_r=&Rr=\begin{pmatrix}
                0 & 0 & 1 & 0 & 0 & 0 & 0 \\
                1 & 0 & 0 & 0 & 1 & 0 & 0 \\
                1 & 0 & 0 & 0 & 0 & 1 & 0 \\
                0 & 0 & 0 & 0 & 0 & 0 & 1
    \end{pmatrix}
    \begin{pmatrix}
        0 \\
        1 \\
        1 \\
        0 \\
        0 \\
        1 \\
        1
    \end{pmatrix}=
    \begin{pmatrix}
        1 \\
        0 \\
        1 \\
        1
    \end{pmatrix}
\end{align*}
