\section{Wektory i wartości własne macierzy; numeryczne algorytmy ich wyznaczania.}

Wektory i wartości własne – wielkości opisujące endomorfizm danej przestrzeni liniowej;
\textbf{wektor własny} przekształcenia można rozumieć jako wektor,
którego kierunek nie ulega zmianie po przekształceniu go endomorfizmem;
\textbf{wartość własna} odpowiadająca temu wektorowi to skala podobieństwa tych wektorów.

\subsection{Klasyczne wyznaczanie wektorów i wartości własnych}

\begin{enumerate}[itemsep=0pt,partopsep=0pt, parsep=0pt]
    \item Szukamy pierwiastków wielomianu
    \[
        w(\lambda)=\det(\mathbf{A}-\lambda\mathbf{I})
    \]
    gdzie $\mathbf{A}$ jest macierzą kwadratową, $\mathbf{I}$ jest macierzą identycznościową,
    a $w(\lambda)$ nazywamy \textbf{wielomianem charakterystycznym}.
    Pierwiastki to \textbf{wartości własne} macierzy; ozn. $\lambda_1$, $\lambda_2$, $\lambda_3$, \ldots
    \item Rozwiązujemy równania
    \[
        (\mathbf{A}-\lambda_i\mathbf{I})\cdot\mathbf{x_i}=0
    \]
    gdzie $\mathbf{x_i}$ to \textbf{wektory własne}.
\end{enumerate}

Uwagi:
\begin{itemize}[itemsep=0pt,partopsep=0pt, parsep=0pt]
    \item Transponowanie nie zmienia wartości własnych macierzy.
    \item Jeśli macierz $A$ jest symetryczna, wszystkie wartości własne są liczbami rzeczywistymi.
\end{itemize}

\subsection{Numeryczne wyznaczanie wektorów i wartości własnych}

\subsubsection{Metoda potęgowa}

Metoda potęgowa w wersji podstawowej służy do wyznaczenia maksymalnej
co do modułu wartości własnej i odpowiadającego jej wektora własnego.

Jeżeli jednak wykorzystamy fakty, że wartości
własne macierzy odwrotnej są odwrotnościami wartości własnych macierzy danej
oraz że modyfikacja macierzy polegająca na dodaniu do elementów jej przekątnej głównej pewnej liczby $d$
powoduje przesunięcie wszystkich wartości własnych tej macierzy o tę samą liczbę $d$ (przesunięcie widma),
możemy przy pomocy metody potęgowej wyznaczyć wartość własną najbliższą dowolnej liczbie.


Rozpoczynając od dowolnego wektora $\overrightarrow{x_0}\in \mathbb{C}^n$
o normie $\lVert \overrightarrow{x_0} \rVert_2 = 1$, obliczamy kolejno dla $k=1,2,\ldots$
\begin{align*}
    \overrightarrow{y_k}=A\overrightarrow{x_k} & & \overrightarrow{x_{k+1}}=\frac{\overrightarrow{y_k}}{\lVert \overrightarrow{y_k} \rVert_2}.
\end{align*}
Równoważnie możemy zapisać
\[
    \overrightarrow{x_k}=\frac{A^k\overrightarrow{x_0}}{\lVert A^k\overrightarrow{x_0} \rVert_2}.
\]
Wektory $x_k$ stanowią kolejne przybliżenia wektora własnego.
Odpowiadającą mu wartość własną przybliżamy na podstawie równania
\begin{gather*}
(\mathbf{A}-\lambda_i\mathbf{I})
    \cdot\mathbf{x_i}=0\\
    \mathbf{A}\mathbf{x_i}=\lambda\mathbf{x_i}\\
\end{gather*}
obliczając iloraz dwóch kolejnych np. ostatnich przybliżeń wektora własnego;
\[
    \lambda=\frac{x_{k-1}^2}{x_k^2}=\frac{Ax_k\cdot x_k}{x_k \cdot x_k}.
\]

\subsubsection{Metoda iteracji odwrotnej (odwrotna metoda potęgowa)}

Wariant metody potęgowej.
Pozwala znaleźć najmniejszą na modół wartość własną.

\subsubsection{Metoda QR}
Nieco bardziej praktyczna dla komputerów.
Metoda iteracyjna.
W kroku k obliczamy QR-dekompozycję $A_k=Q_kR_k$
($Q$ - macierz ortogonalna, czyli jej iloczyn z odwrotnością jest macierzą jedynkową,
$R$ - macierz trójkątna górna). Potem $A_{k+1}=R_kQ_k$.
W końcu macierz $A$ zbiegnie do macierzy trójkątnej, na przekątnej której będą wartości własne.
Kolumny $Q$ natomiast są wektorami własnymi.
