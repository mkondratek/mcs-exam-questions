\section{Wektory i wartości własne macierzy; numeryczne algorytmy ich wyznaczania.}

Wektory i wartości własne – wielkości opisujące endomorfizm danej przestrzeni liniowej;
\textbf{wektor własny} przekształcenia można rozumieć jako wektor,
którego kierunek nie ulega zmianie po przekształceniu go endomorfizmem;
\textbf{wartość własna} odpowiadająca temu wektorowi to skala podobieństwa tych wektorów.

\subsection{Klasyczne wyznaczanie wektorów i wartości własnych}

\begin{enumerate}[itemsep=0pt,partopsep=0pt, parsep=0pt]
    \item Szukamy pierwiastków wielomianu
    \[
        w(\lambda)=\det(\mathbf{A}-\lambda\mathbf{I})
    \]
    gdzie $\mathbf{A}$ jest macierzą kwadratową, $\mathbf{I}$ jest macierzą identycznościową,
    a $w(\lambda)$ nazywamy \textbf{wielomianem charakterystycznym}.
    Pierwiastki to \textbf{wartości własne} macierzy; ozn. $\lambda_1$, $\lambda_2$, $\lambda_3$, \ldots
    \item Rozwiązujemy równania
    \[
        (\mathbf{A}-\lambda_i\mathbf{I})\cdot\mathbf{x_i}=0
    \]
    gdzie $\mathbf{x_i}$ to \textbf{wektory własne}.
\end{enumerate}

Uwagi:
\begin{itemize}[itemsep=0pt,partopsep=0pt, parsep=0pt]
    \item Transponowanie nie zmienia wartości własnych macierzy.
    \item Jeśli macierz $A$ jest symetryczna, wszystkie wartości własne są liczbami rzeczywistymi.
\end{itemize}

\subsection{Numeryczne wyznaczanie wektorów i wartości własnych}

\subsubsection{Metoda potęgowa}

\todo[inline]{opisz to}

\todo[inline]{opisz inne metody}
