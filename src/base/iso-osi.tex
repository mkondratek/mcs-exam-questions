\section{Model ISO OSI. Przykłady protokołów w poszczególnych warstwach.}

Model odniesienia przy omawianiu sieci komputerowych pomagający zrozumieć procesy komunikacyjne zachodzące w sieci.
Składa się z siedmiu warstw.\\

\begin{tabular}{|c|c|c|}
    \hline
    Warstwa OSI  & Jednostka         & Przykład protokołu                  \\
    \hline
    Aplikacji    & Dane              & HTTP, FTP                           \\
    Prezentacji  & Dane              & TLS, MIME, ASCII                    \\
    Sesji        & Dane              & Sockety (nawiązanie sesji TCP), RPC \\
    Transportu   & Segment, datagram & TCP, UDP                            \\
    Sieciowa     & Pakiet            & IPv4, IPv6                          \\
    Łącza danych & Ramka             & PPP, ARP, Ethernet                  \\
    Fizyczna     & Bit, symbol       & Zależne od medium (Bluetooth, WiFi) \\
    \hline
\end{tabular}\\

W praktyce obecnie raczej używa się modelu TCP/IP, który zastępuje górne 3 warstwy jedną warstwą - aplikacji.
