\section{Omówić znane metody synchronizacji procesów.}

Problem synchronizacji procesów pojawia się wszędzie tam, gdzie mamy do czynienia ze współpracującymi ze sobą współbieżnymi procesami.
Oto najczęściej spotykane przyczyny, dla których konieczna jest synchronizacja współpracujących procesów:

\begin{itemize}[itemsep=0pt,partopsep=0pt, parsep=0pt]
    \item Procesy współdzielą pewną strukturę danych —
    Zwykle wykonania operacji na tej strukturze danych nie mogą dowolnie przeplatać się współbieżnie ze sobą.
    Pojedyncze operacje muszą być "zatomizowane",
    tzn. tylko jeden proces może na raz modyfikować współdzieloną strukturę danych,
    stąd konieczność synchronizacji procesów przy dostępie do struktury danych.
    Jest to przykład tzw. problemu sekcji krytycznej.
    \item Wyniki działania jednego procesu stanowią dane dla innego procesu.
    Oczywiście drugi z procesów może przetwarzać dane dopiero wówczas,
    gdy zostaną one obliczone przez pierwszy z procesów, stąd konieczność synchronizacji działań obydwu procesów.
    Dodatkowo, jeżeli dane zajmują dużo pamięci i są obliczane stopniowo, to drugi z procesów może informować pierwszy,
    które dane zostały "zużyte" i można wykorzystać zajmowaną przez nie pamięć.
    Przykładem takiego współdziałania procesów jest problem producenta-konsumenta.
    \item Procesy korzystają z pewnej wspólnej puli zasobów, które pobierają i zwalniają wedle potrzeb.
    Oczekiwanie na zasoby i przydzielanie ich wymaga również synchronizacji między procesami.
    Przykładem dobrze ilustrującym taki rodzaj synchronizacji jest problem pięciu filozofów.
\end{itemize}

Zrealizowanie synchronizacji między procesami wymaga pewnych narzędzi
i technik programistycznych, opisanych w niniejszym wykładzie.
Na pierwszy rzut oka problem może się wydawać trywialny.
Okazuje się jednak, że programowanie procesów współbieżnych jest bardzo trudne.
Liczba zależności między procesami jest trudna do ogarnięcia.

Analizując programy współbieżne zakładamy, że współbieżne wykonanie procesów może polegać
na dowolnym przepleceniu wykonań poszczególnych procesów.
Dodatkowo musimy pamiętać, że przeplatane nie są instrukcje języka programowania wysokiego poziomu,
lecz instrukcje procesora w skompilowanym programie. Jeśli nasz program jest poprawny przy takim założeniu,
to mamy pewność, że będzie działał poprawnie (niezależnie od realizacji systemu operacyjnego,
parametrów technicznych komputera, zachodzących przerwań i innych procesów działających w systemie).

W przypadku programów współbieżnych testowanie programów ma bardzo ograniczone zastosowanie.
Nie jesteśmy w stanie przetestować wszystkich możliwych przeplotów wykonań procesów.
Siłą rzeczy testy są przeprowadzane na konkretnej platformie i konkretnym komputerze.
Tak więc pomyślne wyniki testów wcale nie gwarantują, że program będzie działał poprawnie na innej platformie,
czy np. na szybszym komputerze.
Jesteśmy więc skazani na weryfikowanie poprawności programów współbieżnych.
Już w przypadku programów sekwencyjnych jest to trudne zadanie,
a w przypadku programów współbieżnych jest to bardzo trudne.

\subsection{Problemy}
\begin{itemize}[itemsep=0pt,partopsep=0pt, parsep=0pt]
    \item \textbf{Producenta i konsumenta} -
    Mamy dwa rodzaje procesów: producent i konsument.
    Producent wrzuca dane do bufora, a konsument je konsumuje.
    Nie chcemy dodawać nowych danych, gdy bufor jest pełny, a konsument nie powinien pobierać, gdy bufor jest pusty.
    Rozwiązanie może korzystać np. z dwóch semaforów.
    \item \textbf{Czytelników i pisarzy} -
    Kontrolujemy dostęp do pewnego zasobu (np. plik).
    Zasób może odczytywać jednocześnie dowolna liczba czytelników, ale pisarz musi otrzymać zasób na wyłączność.
    Istnieje kilka możliwości rozwiązania (faworyzujący czytelników, pisarzy, kolejka).
    \item \textbf{Pięciu filozofów} -
    Przy okrągłym stole siedzi 5 filozofów i każdy albo je, albo rozmyśla.
    Przed każdym filozofem jest talerz z jedzeniem, i pomiędzy sąsiednimi filozofami jest widelec.
    Żeby jeść, filozof musi wziąć 2 widelce leżące obok siebie.
    Rozwiązanie to np. kelner (osobny proces zarządzający widelcami),
    albo hierarchia (możemy wziąć tylko najpierw niższy widelec, a potem wyższy, gdy oddajemy jest odwrotnie).
\end{itemize}

\subsection{Metody synchronizacji}
\begin{itemize}[itemsep=0pt,partopsep=0pt, parsep=0pt]

    \item Synchronizacja procesów za pomocą wspólnych zmiennych
    \todo[inline]{opisz to (https://edu.pjwstk.edu.pl/wyklady/sop/scb/wyklad5/wyklad.html)}

    \item Algorytm Dekkera - dla dwóch procesów

    \item Algorytm piekarniany - wydawanie numerków na poczcie

    \item Semafor - Zmienna całkowita przyjmująca wartości nieujemne.
    Proces może opuścić semafor (zmniejszyć wartość o 1) albo podnieść.
    Operacja blokuje się, gdy semafor w danej chwili ma wartość 0.

    \item Mutex - Podobny do semafora o wartości 1.
    Dodatkowym ograniczeniem jest, że proces, który zdejmuje zamek musi być tym samym, który zamek założył.

    \item Kolejka komunikatów (messagebus)

    \item Monitor - abstrakcyjna struktura danych
    zezwalająca na dostęp tylko jednemu procesowi na raz (Java, blok \textit(synchronized))
\end{itemize}
