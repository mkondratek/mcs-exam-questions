\section{Podstawowe metody indeksowania w systemach baz danych.}

\textbf{Indeks} - struktura, która ma przyspieszyć wyszukiwanie w bazie danych.
Indeks definiowany jest dla atrybutów, które nazywamy kluczami indeksu.

\subsection{Typy indeksów}
\begin{itemize}[itemsep=0pt,partopsep=0pt, parsep=0pt]
    \item Indeks podstawowy/główny - zdefiniowany na kluczu podstawowym
    \item Indeks dodatkowy - zdefiniowane na polach, które nie są ani podstawowe, ani nie są porządkujące
    \item Indeks gęsty - indeks, który posiada wpis dla każdej wartości klucza wyszukiwania
    \item Indeks rzadki - indeks, który posiada wpisy tylko dla niektórych wartości
\end{itemize}

\subsection{Metody indeksowania}
\begin{itemize}[itemsep=0pt,partopsep=0pt, parsep=0pt]
    \item Wyszukiwanie binarne i ISAM - za pomocą wyszukiwania binarnego jesteśmy w stanie znaleźć dany klucz
    w posortowanym ciągu w czasie logarytmicznym.
    ISAM modyfikuje ten koncept, dodając dodatkowy poziom - tabelę, gdzie trzymamy indeks "cylindrów"
    (pary klucz - adres cylindra),
    a faktyczne klucze (pary klucz - adres rekordu) są trzymane wewnątrz cylindrów.
    Jeżeli chcemy dodać dane do cylindra, gdzie nie ma już miejsca, tworzymy stronę "nadmiarową",
    więc jeżeli będziemy dodawać dużo rekordów, a mało usuwać, przeszukiwanie stanie się nieefektywne.
    \item B-drzewa - mogą być stosowane do dziedzin, gdzie mamy relację porządkującą.
    B-drzewo jest samobalansującym drzewem (niekoniecznie binarnym), które zapewnia uporządkowanie danych,
    oraz wyszukiwanie, dostęp, wstawianie i usuwanie w czasie logarytmicznym.
    W praktyce zazwyczaj używa się modyfikacji: B+-drzew,
    w których wszystkie indeksowane dane i wskaźniki do rekordów danych przechowywane są w liściach,
    a wierzchołki służą tylko jako pomoc przy dotarciu do właściwego liścia.
    Zmniejsza to ilość koniecznych operacji IO, jakie trzeba wykonać.
    \item Haszowanie - indeksy haszujące przechowują liczbowy hasz na podstawie wartości kolumny.
    Taki indeks potrafi tylko obsłużyć operację równości.
\end{itemize}
