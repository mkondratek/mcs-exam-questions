\section{Dyskretne zmienne losowe oraz ich najważniejsze rozkłady.}

\subsection{Definicje}

\textbf{Przestrzeń zdarzeń elementarnych} - zbiór wszystkich możliwych wyników doświadczenia losowego (np. rzut monetą - orzeł i reszka).

\textbf{Zmienna losowa} - funkcja przypisująca zdarzeniom elementarnym liczby.

\textbf{Zmienna losowa dyskretna} (in. o rozkładzie dyskretnym) to taka zmienna losowa,
która przyjmuje z dodatnim prawdopodobieństwem jedynie skończoną lub nieskończoną przeliczalną liczbę różnych wartości.

\begin{align*}
    \text{Dystrybuanta:} \quad & F(x)=P(X < x)=\sum_{x_i<x}p_i\\
    \text{Wartość oczekiwana:} \quad & \mu = E(X)=\sum x_i p_i\\
    \text{Wariancja:} \quad & \sigma^2 = Var(X)=E\left[ (X - \mu)^2 \right] = \sum (x_i - \mu)^2p_i\\
\end{align*}

\subsection{Najważniejsze rozkałady}

\subsubsection{Rozkład geometryczny}
Opisuje prawdopodobieństwo, że odniesiemy pierwszy sukces w dokładnie $k$-tej próbie,
gdzie każda ma prawdopodobieństwo $p$.
\begin{gather*}
    P(X=k)=(1-p)^{k-1}p\\
    EX=\frac{1}{p}\\
\end{gather*}

\subsubsection{Rozkład dwumianowy}
Opisuje liczbę sukcesów $k$ w ciągu $n$ niezależnych prób, gdzie każda ma prawdopodobieństwo sukcesu równe $p$.
\begin{gather*}
    P(X=k)=\binom{n}{k}p^k(1-p)^{(n-k)}\\
    EX=np\\
\end{gather*}

\subsubsection{Rozkład Poissona}
Opisuje prawdopodobieństwo wystąpienia danej liczby wydarzeń w ustalonym czasie pod warunkiem,
że wydarzenia występują z daną stałą częstotliwością i niezależnie od czasu ostatniego wystąpienia.
(Przykład: call center, gdzie dostajemy średnio 100 połączeń na godzinę, możemy zobaczyć ile dostaniemy w ciągu minuty).
\begin{gather*}
    P(X=k)=\frac{\lambda^ke^{\lambda}}{k!}, \lambda>0\\
    EX=\lambda\\
\end{gather*}

\subsubsection{Rozkład ujemny dwumianowy (Pascala)}
\todo[inline]{opisz to}

\subsubsection{Rozkład jednostajny}
\todo[inline]{opisz to}

\subsubsection{Rozkład dwupunktowy}
\todo[inline]{opisz to}
