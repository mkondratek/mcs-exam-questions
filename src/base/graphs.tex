\section{Graf eulerowski, graf hamiltonowski,
    liczba chromatyczna grafu; definicje i związane z tymi pojęciami twierdzenia.}

\subsection{Graf eulerowski}
\textbf{Graf eulerowski} - da się w nim skonstruować \textbf{cykl Eulera}, czyli cykl,
który przechodzi przez każdą jego krawędź dokładnie raz.

\textbf{Warunek konieczny (tw Eulera):}
Warunkiem koniecznym i wystarczającym na to by spójny graf nieskierowany
był eulerowski jest parzystość stopni wszystkich wierzchołków.
Natomiast warunkiem w spójnym grafie skierowanym
jest taka sama liczba krawędzi wchodzących i wychodzących dla każdego wierzchołka.

\subsection{Graf hamiltonowski}
\textbf{Graf hamiltonowski} - rodzaj grafu rozważany w teorii grafów i definiowany dwojako, w dwóch nieco innych znaczeniach:
\begin{itemize}[itemsep=0pt,partopsep=0pt, parsep=0pt]
    \item szerszym: dowolny graf zawierający ścieżkę (drogę) przechodzącą przez
    każdy wierzchołek dokładnie jeden raz zwaną \textbf{ścieżką Hamiltona};
    \item węższym: grafem hamiltonowskim jest graf zawierający \textbf{cykl Hamiltona}, tj. zamkniętą ścieżkę Hamiltona.
\end{itemize}

\textbf{Warunek konieczny:} Jeżeli graf $G$ jest hamiltonowski
to dla każdego niepustego podzbioru $V^{\prime}$ zbioru wierzchołków $V(G)$ zachodzi
\[
    \omega(V(G) - V^{\prime}) \leqslant \left| V^{\prime} \right|
\]
gdzie $\omega (G)$ oznacza liczbę spójnych składowych grafu $G$.


\textbf{Twierdzenie Ore:} weźmy graf spójny o liczbie wierzchołków $n \geq 3$.
Jeżeli $deg(u)+deg(w)\geq n$ dla dowolnej pary wierzchołków $u$, $w$
które nie są połączone krawędzią to graf $G$ jest grafem hamiltonowskim.

\subsection{Liczba chromatyczna}

\textbf{Liczba chromatyczna} – liczba kolorów niezbędna do optymalnego klasycznego (wierzchołkowego) pokolorowania grafu,
czyli najmniejsza możliwa liczba $k$ taka, że możliwe jest legalne
(wierzchołki o tych samych kolorach nie sąsiadują ze sobą)
pokolorowanie wierzchołków grafu $k$ kolorami.
Oznacza się ją symbolem $\chi (G)$.

Problem wyznaczenia liczby chromatycznej jest NP-trudny
– nie są znane niezawodne wielomianowe algorytmy wyznaczające liczbę chromatyczną każdego grafu.
Istnieje jednak szereg oszacowań liczby chromatycznej dla różnych klas grafów, np.:

\begin{itemize}[itemsep=0pt,partopsep=0pt, parsep=0pt]
    \item $\chi (G)\geqslant \omega$, gdzie $\omega$  jest rozmiarem maksymalnej kliki grafu $G$,
    \item Twierdzenie Brooksa: dla grafów pełnych oraz cykli o nieparzystej długości $\chi (G)=\Delta +1$,
    gdzie $\Delta$  jest maksymalnym stopniem wierzchołka w grafie $G$;
    \item dla pozostałych grafów spójnych zachodzi $ \chi (G)\leqslant \Delta$,
    \item dla grafów planarnych  $\chi (G)\leqslant 4$, dla drzew o co najmniej dwóch wierzchołkach $\chi (G)=2$.
\end{itemize}
