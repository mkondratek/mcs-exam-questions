\section{Przedstawić i omówić hierarchię Chomsky’ego
ze szczególnym uwzględnieniem występujących w niej definicji gramatyk.}

\textbf{Hierarchia Chomsky’ego} to hierarchia języków formalnych, składająca się z czterech klas:
\begin{itemize}[itemsep=0pt,partopsep=0pt, parsep=0pt]
    \item typ 3 - regularne
    \item typ 2 - bezkontekstowe
    \item typ 1 - kontekstowe
    \item typ 0 - rekurencyjnie przeliczalne.
\end{itemize}
Gramatyka to sposób opisu języka.
Składa się ze zbioru terminali $\sum$, zbioru nieterminali $N$,
symbolu startowego $S$ i zbioru reguł przepisywania $\alpha\longrightarrow\beta$.

Na samym początku forma zdaniowa będzie się składać tylko z symbolu startowego.
W każdym kroku możemy przekształcić formę zdaniową korzystając z reguł przepisywania.
Do języka należą wszystkie słowa nad alfabetem terminali, które da się w ten sposób uzyskać.

Każda gramatyka dopuszcza wszystkie reguły gramatyk o wyższym numerze (klasy się zawierają przez zawieranie ostre).

\subsection{Język regularny}
Gramatyka liniowa, czyli taka,
która przekształca jeden nieterminal w słowo zawierająca ciąg terminali i co najwyżej jeden nieterminal.
Równoważne automatom skończonym.

\subsection{Język bezkontekstowy}
Gramatyka bezkontekstowa, czyli posiadająca tylko reguły postaci $A\longrightarrow\gamma$, gdzie $A$ jest nieterminalem,
a $\gamma$ jest dowolnym słowem.
Równoważne niedeterministycznym automatom ze stosem.

\subsection{Język kontekstowy}
Gramatyka kontekstowa, czyli posiadająca reguły postaci $\alpha A\beta\longrightarrow\alpha\gamma\beta$,
gdzie $A$ to nieterminal, $\alpha, \beta$ to dowolne słowa, a $\gamma$ to dowolne niepuste słowa.
Równoważne automatom ograniczonym liniowo.

\subsection{Język rekurencyjnie przeliczalne}
Gramatyka \textit{typu 0}, czyli reguły postaci $\alpha\longrightarrow\beta$ (dowolne).
Równoważne maszynom Turinga.
