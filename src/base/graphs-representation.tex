\section{Omówić metody reprezentacji grafów oraz metody przeglądania (DFS, BFS).}

\subsection{Metody reprezentacji grafów}

\subsubsection{Lista krawędzi}
Złożoność pamięciowa $\Theta(E)$.
Sprawdzenie czy istnieje krawędź $(i, j)$ jest liniowe, o ile nie ma żadnego porządku na krawędziach.

\subsubsection{Macierz sąsiedztwa}
Złożoność pamięciowa $\Theta(V^2)$.
Sprawdzenie czy istnieje krawędź $(i, j)$ jest w czasie stałym.
Jeśli graf jest nieskierowany to macierz jest symetryczna.
Jeśli jest mało krawędzi (graf jest rzadki) to macierz zawiera głównie zera.

\subsubsection{Lista sądziedztwa}

Złożoność pamięciowa $\Theta(2E)$.
Sprawdzenie czy istnieje krawędź $(i, j)$ jest w czasie liniowym ze względu na $d$ (stopień wierzchołka $i$).

\subsection{Metody przeglądania}

\subsubsection{Depth-first search}

Przeszukiwanie w głąb polega na badaniu wszystkich krawędzi wychodzących z podanego wierzchołka.
Po zbadaniu wszystkich krawędzi wychodzących z danego wierzchołka algorytm powraca do wierzchołka,
z którego dany wierzchołek został odwiedzony. \\

\begin{samepage}
    \begin{verbatim}
    function VisitNode(u):
        oznacz u jako odwiedzony
        dla każdego wierzchołka v na liście sąsiedztwa u:
            jeżeli v nieodwiedzony:
                VisitNode(v)
    function DepthFirstSearch(Graf G):
        dla każdego wierzchołka u z grafu G:
            oznacz u jako nieodwiedzony
        dla każdego wierzchołka u z grafu G:
            jeżeli u nieodwiedzony:
        VisitNode(u)
    \end{verbatim}
\end{samepage}

Wyjściem algorytmu jest pewne drzewo rozpinające przeszukiwany graf.
Pozwala na znajdowanie spójnych składowych grafu, sortowanie topologiczne, znajdowanie mostów, znajdowanie cykli w grafie.

\subsubsection{Breadth-first search}

Przechodzenie grafu rozpoczyna się od zadanego wierzchołka $s$ i polega na odwiedzeniu wszystkich osiągalnych z niego wierzchołków.
Wynikiem działania algorytmu jest drzewo przeszukiwania wszerz o korzeniu w $s$, zawierające wszystkie wierzchołki osiągalne z $s$.
Do każdego z tych wierzchołków prowadzi dokładnie jedna ścieżka z $s$, która jest jednocześnie najkrótszą ścieżką w grafie wejściowym.
Algorytm działa prawidłowo zarówno dla grafów skierowanych jak i nieskierowanych. \\

\begin{samepage}
    \begin{verbatim}
    funkcja BreadthFirstSearch (Graf G, Wierzchołek s)
        dla każdego wierzchołka u z G:
            kolor[u] = biały
            odleglosc[u] = inf
            rodzic[u] = NIL
        kolor[s] = SZARY
        odleglosc[s] = 0
        rodzic[s] = NIL
        Q.push(s)
        dopóki kolejka Q nie jest pusta:
            u = Q.front()
            Q.pop()
            dla każdego v z listy sąsiedztwa u:
                jeżeli v jest biały:
                    kolor[v] = SZARY
                    odleglosc[v] = odleglosc[u] + 1
                    rodzic[v] = u
                    Q.push(v)
            kolor[u] = CZARNY
    \end{verbatim}
\end{samepage}

BFS pozwala na znalezienie najkrótszej ścieżki do danego wierzchołka.
