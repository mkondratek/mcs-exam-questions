\section{Omówić pojęcia przypadku użycia systemu i scenariusza przypadku użycia.}

\subsection{Przypadek uzycia}

\textbf{Przypadki użycia} (use case) są opisami sekwencji interakcji, jakie zachodzą między systemem a aktorem,
gdzie ważne jest, by aktor osiągnął wyznaczony cel, jak np. zmienił dane profilowe w sklepie internetowym.

\subsection{Scenariusz przypadku użycia}

\textbf{Scenariusze przypadków użycia} to słowne opisy postępowania dla danego przypadku
(konkretne ścieżki, happy path, wątki alternatywne).
Swoisty algorytm przedstawiony np. jako ponumerowana lista kroków.
Tak jak przypadek użycia można opisać za pomocą cech:
\begin{itemize}[itemsep=0pt,partopsep=0pt, parsep=0pt]
    \item Nazwa
    \item Opis
    \item Przepływ zdarzeń (scenariusze)
    \item Zależności i relacje
    \item Diagramy aktywności
    \item Wymagania specjalne
    \item Warunki wstępne
    \item Warunki końcowe
\end{itemize}

\subsection{Przykład}

\noindent Nazwa: Dokonaj rezerwacji\\
Inicjator: Rezerwujący\\
Cel: Zarezerwować pokój w hotelu\\
Główny scenariusz:
\begin{enumerate}[itemsep=0pt,partopsep=0pt, parsep=0pt]
    \item Rezerwujący zgłasza chęć dokonania rezerwacji
    \item Rezerwujący wybiera hotel, datę, typ pokoju
    \item System podaje cenę pokoju
    \item Rezerwujący prosi o rezerwację
    \item Rezerwujący podaje swoje potrzebne dane
    \item System dokonuje rezerwacji i nadaje jej identyfikator
    \item System podaje Rezerwującemu identyfikator rezerwacji i przesyła go mailem
\end{enumerate}
Rozszerzenia:
\begin{enumerate}[itemsep=0pt,partopsep=0pt, parsep=0pt]
    \item Pokój niedostępny.
    \begin{enumerate}[itemsep=0pt,partopsep=0pt, parsep=0pt]
        \item System przedstawia inne możliwości wyboru
        \item Rezerwujący dokonuje wyboru
    \end{enumerate}
    \item Rezerwujący odrzuca podane możliwości
    \begin{enumerate}[itemsep=0pt,partopsep=0pt, parsep=0pt]
        \item Niepowodzenie
    \end{enumerate}
\end{enumerate}
