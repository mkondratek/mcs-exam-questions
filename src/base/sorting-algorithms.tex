\section{Omówić znane szybkie algorytmy sortowania przez porównanie.}

\subsection{Mergesort}
Dzielimy zbiór na 2 równe części.
Sortujemy każdą z części osobno.
Łączymy części w jeden posortowany ciąg.

\subsection{Heapsort}
Używamy kopca binarnego.
Kopiec binarny to struktura, gdzie dla każdego wierzchołka, każde jego dziecko jest mniejsze.
Dzieci wierzchołka i mamy pod indeksami $2i$ oraz $2i+1$.
Nowy wierzchołek dodajmy poprzez wrzucenie go na koniec listy, a potem przepychanie w górę,
aż do przywrócenia warunku kopca.
Usuwanie wierzchołka polega na wzięciu ostatniego wierzchołka na szczyt i przepychaniu go w dół.

Z elementów tworzymy kopiec binarny.
Usuwamy po kolei wszystkie elementy kopca, konstruując posortowaną tablicę.

\subsection{Quicksort}
Z tablicy wybieramy (np. losowo) element rozdzielający, co dzieli nam tablicę na dwa fragmenty:
mniejsze niż pivot i większe niż pivot.
Sortujemy rekurencyjnie obydwie części.

\subsection{Porównanie}
\begin{tabular}{|c|c|c|c|}
    \hline
    & Mergesort    & Heapsort     & Quicksort                     \\
    \hline
    Zł. czasowa & $O(n\log n)$ & $O(n\log n)$ & $O(n \log n)$ ($\Theta(n^2)$) \\
    Stabilność  & tak          & nie          & nie                           \\
    \hline
\end{tabular}
