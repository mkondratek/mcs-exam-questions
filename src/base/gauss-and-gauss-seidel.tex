\section{Rozwiązywanie układów równań liniowych: metoda eliminacji Gaussa i metoda Gaussa-Seidla.}

\subsection{Układ rówań liniowych}

Równanie postaci $Ax=b$, gdzie $A$ to macierz wielkości $[m x n]$, $x$ i $b$ to wektory $[m x 1]$.
$m$ to liczba równań, a $n$ to liczba niewiadomych.
Można przekształcić układ równań w inny, który ma ten sam zbiór rozwiązań.
Robimy to poprzez operacje elementarne:
\begin{itemize}[itemsep=0pt,partopsep=0pt, parsep=0pt]
    \item dodanie do równania innego równania,
    \item zamiana dwóch równań miejscami,
    \item pomnożenie równania przez liczbę.
\end{itemize}

\textbf{Eliminacja Gaussa}
Sprowadzamy macierz rozszerzoną układu równań do postaci schodkowej.
Inaczej mówiąc, bierzemy pierwszy wiersz i odejmujemy go od wszystkich niżej, tak żeby wyzerować kolumnę.
Postępujemy analogicznie dla kolejnych wierszy.
Następnie odczytujemy rząd macierzy (liczbę wierszy niezerowych).
Jeżeli jest równa liczbowi wierszy, to mamy jedno rozwiązanie.
Jak mniej to nieskończenie wiele rozwiązań.

\textbf{Metoda Gaussa-Seidla}
Metodę tę możemy zastosować, gdy macierz jest przekątniowo dominująca - oznacza to,
że suma wartości bezwzględnych elementów na diagonali jest większa niż suma poza diagonalą.
Jeżeli tak nie jest, to możemy chcieć przestawić elementy tak, żeby było dobrze.

Najpierw wyznaczamy początkowe wartości zmiennych, np. same zera.
Wyliczamy wartości $x_1,\ldots, x_n$ w zależności od wszystkich innych.
Potem iteracyjnie obliczamy po kolei $x_1, x_2, \ldots$ korzystając już z policzonych wartości.
Kończymy, gdy zmiany są niewielkie.
