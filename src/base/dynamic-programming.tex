\section{Na czym polega programowanie dynamiczne? Własności i przykłady zastosowań.}


Technika projektowania algorytmów polegająca na podziale problemu na mniejsze podproblemy względem kilku parametrów.
Problem musi mieć pewne podzadania - tak, że metoda dziel i zwyciężaj rozwiązywałaby te same podzadania wielokrotnie.
By tego uniknąć, stosujemy technikę spamiętywania.

\subsection{Własność optymalnej podstruktury}
Dla problemów programowania dynamicznego musi zachodzić własność optymalnej podstruktury.
Oznacza to, że optymalne rozwiązanie problemu jest funkcją optymalnych rozwiązań podproblemów,
czyli znając ich optymalne rozwiązania, można efektywnie wyznaczyć rozwiązanie problemu.

\subsection{Metody}
\textbf{Metoda zstępująca} z zapamiętywaniem polega na rekurencyjnym wywołaniu funkcji z wywołaniem wyników.
Praktycznie taka sama, jak dziel i zwyciężaj, z tą różnicą, że pamiętamy już policzone rozwiązania.
\textbf{Metoda wstępująca} polega na rozwiązywaniu wszystkich możliwych podproblemów zaczynając od tych o najmniejszym rozmiarze.
Potencjalnie zużywamy mniej pamięci, ale może się zdarzyć, że część problemów rozwiążemy nadmiarowo.

\subsection{Przykład}
\begin{enumerate}[itemsep=0pt,partopsep=0pt, parsep=0pt]
    \item Obliczanie ciągu Fibonacciego możemy zrealizować za pomocą programowania dynamicznego.
    W liście trzymamy rozwiązania - początkowo $[0, 1]$.
    Potem $n-2$ razy dodajemy na koniec listy sumę dwóch ostatnich elementów.

    \item Mamy planszę n x m. Na niektórych polach są diamenty.
    Idziemy z $(n, m)$ i chcemy dojść na $(1, 1)$ zbierając jak najwięcej diamentów,
    ale możemy iść tylko w lewo lub do góry.

    \item Najdłuższy wspólny podciąg (LCS)

    \item Problem plecakowy - definiujemy $A(i, j)$ jako największą możliwą wartość,
    która może być otrzymana przy założeniu wagi mniejszej lub równej $i$, oraz pierwszych $j$ elementów.
\end{enumerate}