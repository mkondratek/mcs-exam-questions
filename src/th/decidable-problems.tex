\section{Obliczalność i złożoność - Problemy rozstrzygalne, częściowo rozstrzygalne i nierozstrzygalne;
podać przykład problemów z każdej z tych klas. Podać dowód nierozstrzygalności przykładowego problemu. }

\subsection{Definicje}
\textbf{Problem decyzyjny} - pytanie sformułowane w systemie formalnym, na które są możliwe dwie odpowiedzi: tak lub nie.
Problem decyzyjny jest równoważny językowi formalnemu poprawnych rozwiązań.

\textbf{Problem rozstrzygalny} - taki, dla którego istnieje algorytm,
który dla dwolnych danych po skończonej liczbie kroków da rozwiązanie problemu.
Przykładem takiego problemu jest problem "czy dana liczba naturalna jest pierwsza"
- możemy sprawdzić wszystkie jej dzielniki w skończonym czasie.

\textbf{Problem nierozstrzygalny} - taki, dla którego powyższy algorytm nie istnieje.
Przykładem takiego problemu jest problem stopu, a takkże odwrotność problemu stopu.
Problem częściowo rozstrzygalny - gdy istnieje algorytm (maszyna Turinga bez własności stopu),
który poprawnie odpowiada tylko tak, natomiast nie istnieje algorytm poprawnie udzielający odpowiedzi i tak, i nie.
Przykładem jest problem stopu (ale nie odwrotność problemu stopu).

\subsection{Dowód nierozstrzygalności problemu stopu}
\textbf{Problem stopu} - dla danego algorytmu i danych wejściowych odpowiada na pytanie,
czy program zakończy się w skończonym czasie.

Weźmy stop(program), który dla danego programu mówi, czy program się kończy, czy też nie.
Zróbmy na jego podstawie złośliwą funkcję test(), która jeżeli stop(test), to zapętl\_się().

Jeżeli założymy, że stop(test) jest rozstrzygalny, to musi się zakończyć odpowiedzią tak lub nie.
Jeżeli zwróci true, to zapętlamy się i nigdy nie wychodzimy, co jest sprzecznością.
Jeżeli zwróci false, to my wychodzimy, czyli powinniśmy zwrócić true, co również jest sprzecznością.
Doszliśmy do sprzeczności - problem stopu nie jest rozstrzygalny.
