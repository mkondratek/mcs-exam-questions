\section{Obliczalność i złożoność - Definicja złożoności obliczeniowej.
Klasy złożoności: P, NP, PSPACE,NSPACE; zupełność w klasie.
Podać dowód NP*zupełności przykładowego problemu.}

\subsection{Czasowa złożoność obliczeniowa}
Przyjętą miarą złożoności czasowej jest liczba operacji podstawowych w zależności od rozmiaru wejścia.
Pomiar rzeczywistego czasu zegarowego jest mało użyteczny ze względu na silną zależność od sposobu realizacji algorytmu,
użytego kompilatora oraz maszyny, na której algorytm wykonujemy.
Dlatego w charakterze czasu wykonania rozpatruje się zwykle liczbę operacji podstawowych (dominujących).
Operacjami podstawowymi mogą być na przykład: podstawienie, porównanie lub prosta operacja arytmetyczna.

\subsection{Pamięciowa złożoność obliczeniowa}
Podobnie jak złożoność czasowa jest miarą czasu działania algorytmu,
tak złożoność pamięciowa jest miarą ilości wykorzystanej pamięci.
Jako tę ilość najczęściej przyjmuje się użytą pamięć maszyny abstrakcyjnej
(na przykład liczbę komórek pamięci maszyny RAM) w funkcji rozmiaru wejścia.
Możliwe jest również obliczanie rozmiaru potrzebnej pamięci fizycznej wyrażonej w bitach lub bajtach.

\subsection{Definicje}

\textbf{Problem $P$} - problem decyzyjny, dla którego rozwiązanie można znaleźć w czasie wielomianowym.

\textbf{Problem $NP$} - problem decyzyjny, dla którego rozwiązanie można zweryfikować w czasie wielomianowym.

\textbf{$PSPACE$} - zbiór wszystkich problemów decyzyjnych,
które można rozwiązać za pomocą maszyny Turinga wykorzystującej wielomianową przestrzeń.

\textbf{$NPSPACE$} - zbiór wszystkich problemów decyzyjnych,
które można rozwiązać za pomocą niedeterministycznej maszyny Turinga wykorzystującej przestrzeń.

% https://stackoverflow.com/questions/1857244/what-are-the-differences-between-np-np-complete-and-np-hard
% Porblem NP-trudny nie musi być decyzyjny (problem stopu); problem NP-zupełny można zredukować do NP w czasie wielomianowym

Pytanie, czy klasa $P$ jest tym samym co $NP$, jest jednym z problemów milenijnych,
za których rozwiązanie przewidziano nagrodę w wysokości 1 miliona dolarów.
Każdy problem z klasy $P$ jest również w klasie $NP$,
nie wiadomo jednak czy istnieją problemy klasy $NP$, które nie są problemami klasy $P$.

\subsection{Dowód NP*zupełności problemu 4-SAT}

\subsubsection{4-SAT jest $NP$}
Problem jest $NP$, jeśli możemy zweryfikować jego rozwiązanie w czasie wielomianowym.
Mając dane przypisanie zmiennych, możemy w liniowy sposób przejść po formule i sprawdzić jej wartościowanie.

\subsubsection{4-SAT jest $NP$-trudny}

Aby pokazać, że 4-SAT jest $NP$-trudny, zredukujemy go do znanego problemu $NP$-trudnego.
Dokonamy redukcji z 3-SAT do 4-SAT.
Do każdej klauzuli $c$ formuły problemu 3-SAT dodajemy literał $a$ lub $\bar{a}$.
Niech $c=x \lor y \lor z$;
zamieniamy $c$ na $c'=(x \lor y \lor z \lor a) \land (x \lor y \lor z \lor \bar{a})$.
Po dokonaniu tej transformacji zachodzą:
\begin{enumerate}[itemsep=0pt,partopsep=0pt, parsep=0pt]
    \item jeśli początkowy 3-SAT ma spełniające wartościowanie, to skonstruowany 4-SAT również je ma (to samo!),
    ponieważ zbiór klauzul jest złożony ze zmiennej i jej negacji - nie robi różnicy.
    \item jeśli skonstruowany 4-SAT jest spełniony dla jakiegoś,
    $(x \lor y \lor z \lor a) \land (x \lor y \lor z \lor \bar{a})$, to początkowy 3-SAT również jest spełniony,
    ponieważ $a$ i $\bar{a}$ się dopełniają, zatem formuła musi być spełniona przez jakieś inny literał prócz $a$.
\end{enumerate}

4-SAT jest $NP$*zupełny.
