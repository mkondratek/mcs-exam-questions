\section{Kodowanie informacji - Kody prefiksowe, kodowanie Huffmana.}

\subsection{Kod prefiksowy}

Kod prefiksowy lub przedrostkowy (ang. prefix code)
– kod, w którym żadne ze słów kodowych nie jest przedrostkiem innego słowa;
taki kod jest jednoznacznie dekodowalny.
Dodatkowo każdy kod prefiksowy można reprezentować w formie drzewa (dla kodów dwójkowych to drzewo binarne).

Dzięki tej cesze kody są jednoznacznie identyfikowane,
nie ma potrzeby wstawiania dodatkowych informacji np. o tym, gdzie kończy się słowo kodowe (jest to jednoznaczne)
albo jaką ma długość (długość każdego słowa kodowego jest znana z góry).
Stosując kody prefiksowe, można uzyskać maksymalny stopień upakowania danych w różnych metodach kompresji.

Dla przykładu weźmy kod niebędący prefiksowym:
\begin{itemize}[itemsep=0pt,partopsep=0pt, parsep=0pt]
    \item literze $a$ odpowiada bit $0$,
    \item literze $b$ odpowiada bit $1$,
    \item zaś literze $c$ dwa bity $01$
\end{itemize}
– kod litery $a$ jest prefiksem kodu litery $c$.
Przy takim przyporządkowaniu nie można jednoznacznie stwierdzić,
co oznacza np. komunikat $0110$ – może to być zarówno $cba$, jak i $abba$.

Zmieniając kod na prefiksowy:
\begin{itemize}[itemsep=0pt,partopsep=0pt, parsep=0pt]
    \item $a$ – $0$,
    \item $b$ – $10$,
    \item $c$ – $11$,
\end{itemize}
ten sam komunikat ma jednoznaczną interpretację, tj. $aca$.

\subsection{Kodowanie Huffmana}
Kodowanie Huffmana to jedna z najprostszych i łatwych w implementacji metod kompresji bezstratnej.

Algorytm Huffmana nie należy do najefektywniejszych obliczeniowo systemów bezstratnej kompresji danych,
dlatego też praktycznie nie używa się go samodzielnie.
Często wykorzystuje się go jako ostatni etap w różnych systemach kompresji,
zarówno bezstratnej, jak i stratnej, np. MP3 lub JPEG.
Pomimo że nie jest doskonały, stosuje się go ze względu na prostotę oraz brak ograniczeń patentowych.
Jest to przykład wykorzystania algorytmu zachłannego.

Dany jest alfabet źródłowy (zbiór symboli) $S=\{x_{1}, \dots, x_{n}\}$
oraz zbiór stowarzyszonych z nim prawdopodobieństw $P=\{p_{1}, \dots, p_{n}\}$.
Symbolami są najczęściej bajty, choć nie ma żadnych przeszkód, żeby było nimi coś innego (np. pary znaków).
Prawdopodobieństwa mogą zostać z góry określone dla danego zestawu danych,
np. poprzez wyznaczenie częstotliwości występowania znaków w tekstach danego języka.
Częściej jednak wyznacza się je indywidualnie dla każdego zestawu danych.

Własności kodu Huffmana są następujące:
\begin{itemize}[itemsep=0pt,partopsep=0pt, parsep=0pt]
    \item jest kodem prefiksowym; oznacza to, że żadne słowo kodowe nie jest początkiem innego słowa;
    \item średnia długość słowa kodowego jest najmniejsza spośród kodów prefiksowych;
    \item jeśli prawdopodobieństwa są różne, tzn. $p_{j}>p_{i}$,
    to długość kodu dla symbolu $x_{j}$ jest nie większa od kodu dla symbolu $x_{i}$;
    \item słowa kodu dwóch najmniej prawdopodobnych symboli mają równą długość.
\end{itemize}
Kompresja polega na zastąpieniu symboli otrzymanymi kodami.
