\section{Modelowanie systemów liczących -
Twierdzenie Little'a i jego zastosowanie w analizie modeli systemów kolejkowych systemów.
Szkic dowodu twierdzenia Little'a.}

\textbf{Prawo Little’a} – twierdzenie mówiące o tym,
że średnia liczba rzeczy/klientów w systemie jest równa iloczynowi średniego czasu przebywania w systemie
oraz średniego tempa ich przybywania.

Prawo to jest elementem teorii kolejek – dziedziny matematyki będącej częścią badań operacyjnych.

Prawo Little’a jest wyrażone wzorem:
\[
    L = \lambda W
\]
\begin{align*}
    L = & \text{średnia liczba rzeczy/klientów w systemie/kolejce;}\\
    \lambda = & \text{średnie tempo przybywania (intensywność napływu zgłoszeń);}\\
    W = & \text{średni czas przebywania w systemie}
\end{align*}

Przykład: jeżeli oddział banku odwiedza średnio 10 klientów na godzinę ($\lambda$)
i klient przebywa w nim średnio 1 godzinę ($W$),
to średnia liczba klientów znajdujących się w oddziale ($L$) wynosi 10.

\subsection{Szkic dowodu (FIFO)}

\[
    N(\tau)=\alpha(\tau)-\beta(\tau)
\]
gdzie $N(\tau)$ to liczba klientów w systemie w chwili $\tau$,
$\alpha(\tau)$ to liczba klientów, którzy przyszli w czasie $[0, \tau]$,
$\beta(\tau)$ to liczba klientów, którzy poszli w czasie $[0, \tau]$.

\begin{gather*}
    A(t)=\int_0^t N(\tau) d\tau\\
    A(t)=\sum_{t=1}^{\alpha(t)} T(i)\\
    \frac{1}{t}\int_0^t N(\tau) d\tau=\frac{1}{t}\sum_{t=1}^{\alpha(t)} T(i)=\frac{\alpha(t)}{t}\frac{\sum_{t=1}^{\alpha(t)} T(i)}{\alpha(t)}\\
\end{gather*}

% https://www.richardclegg.org/sites/default/files/Lecture4_05.pdf
